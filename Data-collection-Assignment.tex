
\documentclass{article}                    % article class

\begin{document}
\begin{titlepage} % Suppresses headers and footers on the title page

	\centering % Centre everything on the title page
	
	\scshape % Use small caps for all text on the title page
	
	\vspace*{\baselineskip} % White space at the top of the page
	
	%------------------------------------------------
	%	Title
	%------------------------------------------------
	
		
	\vspace{0.80\baselineskip} % Whitespace above the title
	
	{\LARGE MAKERERE UNIVERSITY\\ COLLEGE OF COMPUTING AND INFORMATION SCIENCES\\ DEPARTMENT OF COMPUTER SCIENCE\\BIT2207: RESEARCH METHODOLOGY\\LECTURER: MR ERNEST MWEBAZE\\} % Title
	
	\vspace{8.00\baselineskip} % Whitespace below the title
	

	
	%------------------------------------------------
	%	Student
	%------------------------------------------------
	
	\vspace{0.5\baselineskip} % Whitespace before the student details
	
	{\scshape\Large NAME:KAMIKAZI LINDA\\REG. NO:16/U/5329/EVE\\STUDENT NO.216014116 \\} %student details
	
	\vspace{0.5\baselineskip} % Whitespace below the student details
\end{titlepage}
\newpage
{\textbf{UNEMPLOYMENT IN YOUTH  URBAN AREAS OF UGANDA}}
\section{Abstract}
Unemployment refers to a situation where people in a working age group are available for paid employment or self-employment but there is no available work for them to do. 
\section{Introduction:  }
The range of working age in Uganda is between 15-64years. Uganda’s national unemployment rate stood at 11.7  percent in October 2012. The rural unemployment rate was 1.7percent. Urban unemployment rate for the youth stands at 12 percent and unemployment rate for youth was 32.2 percent  in Kampala in 2010. Youth refers to people aged between 12 and 30 years
\section{Literature Review}
Uganda statistics from the labour department indicate that 390,000 students who finish tertiary education each year have only 8,000 jobs to fight for. This means that for every one job that is available they are about 50 people to fill it.

The labour force flow figures at the Uganda Investment Authority (UIA) and the Uganda Bureau of Statistics (UBOS) indicate  more than  400,000 Ugandans who enter the labour market each year, only about 113,000 are absorbed in formal employment, leaving the rest have to join the informal sector. The UBOS findings indicate that illiterates are more likely to be available for any work than the literates.


 
Uganda’s unemployment rate stands at 80 per cent and underemployment, which is mainly prevalent in rural areas at 17 per cent. Statistics from the Labour Department show that the current labour force is estimated at 9.8 million of which 53 per cent are females.
\section{Problem statement  }
A high level of youth unemployment is one of the critical socio-economic problems facing Uganda (MoGLSD 2012). While the labor force grows, with an increasing proportion of youth, employment growth is inadequate to absorb labor market entrants. As a result, youth are especially affected by unemployment. Moreover, young people are more likely to be employed in jobs of low quality, working long hours for low wages, engaged in dangerous work or receive only short term and/or informal employment arrangements (Haftendorn and Salzano 2004). Despite the efforts put in, high rates of unemployment continue to occur in the districts, youth with formal education had low probabilities of being employed and their unemployment rate stood at 18 percent compared to the 13.3 percent who had completed some stage of education. Various measures have been taken by the government of Uganda to alleviate the problem of youth unemployment, 3 million Ugandan shillings was provided to the directorate of industrial training to train youth in skills that will make them more employable (UG budget FY 2009/10). Therefore, there has not been enough extensive research made in the area of study so, this has  prompted the researcher to find the impact of education level, level of capital and dependence  burden of youth unemployment in urban areas of Uganda.
\section{Scope}
This research is aimed at identifying:
\vspace{1.08\baselineskip}
The study was carried out in urban areas of Uganda; the study was mainly focused on the investigation of determinants of youth unemployment with much emphasis in education level, level of capital and the dependence burden.
\subsection{Objectives}
The general objective is to examine the major determinates of youth unemployment in youth urban areas of Uganda.
But for matters of research convenience there are specfic objectives that can be pointed out such as;
\vspace{1.08\baselineskip}
{\newline}
1. To examine the impact of education and training level on youth unemployment.
\vspace{1.08\baselineskip}
{\newline}
2. To examine the relationship between the level of capital and youth unemployment.
\vspace{1.08\baselineskip}
{\newline}
3. To determine the impact of dependence burden on youth unemployment.
\section{Methodology}
This section includes the study area, research design, data type and sources, sampling selection and size, data collection methods and data analysis techniques. 
\vspace{2.00\baselineskip}
\subsection{Research significance}
The will be useful to the researcher through acquiring skills of doing research and also to put theoretical knowledge into practical experience. The study will also help the researcher to interact with many different categories of people which will help her in discovering important data about the youth unemployment. Through this study, the research will learn different methods of collecting and analyzing data.
\section{CONCLUSION:}
The study shows that unemployment among the youth in urban ares of Uganda is a big issue at hand that should be worked on with effect cause people's lifestyles are at stake and that of the countries economy.
\newpage
\section{Reference:}
1.fortuneofafrica.com
\vspace{2.00\baselineskip}
{\newline}
2.http://www.academia.edu

\end{document}